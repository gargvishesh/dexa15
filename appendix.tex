\newpage
\section{Appendix}

\subsection{DRAM size sensitivity analysis}
In this section, we cover the scenario wherein the available DRAM size
at runtime is less than that anticipated prior to the query execution --
for example, due to concurrent query processing.  To model this scenario,
we experiment with DRAM sizes of 512 KB, 1 MB and 2 MB. Note that, for
each of these cases, all the algorithms continue to assume 4 MB as the
available DRAM size, oblivious to the actual reduction in the availability
of DRAM, and hence continue to execute accordingly. The results are shown
in Figures~\ref{fig:sort}, \ref{fig:join} and \ref{fig:gby} for the Sort,
Hash Join and Group By operators, respectively.

In Figure~\ref{fig:sort}, we observe that the Sort operator provides average savings
for Q13 of about 47\% in Writes and 20\% in Cycles.

\begin{figure}[h]
	\centering
	\includegraphics[height=29mm]{sort_q13.png}
    \caption{Sort (Q13)}
	\label{fig:sort}
\end{figure}






%\subsubsection{Sort}

\begin{figure}[h]
	\centering
	\subfloat[Q16]{	
  	\includegraphics[height=29mm]{hj_q16.png}
	}
%	\hspace{0mm}
	\subfloat[Q19]{
  	\includegraphics[height=29mm]{hj_q19.png}
	}
	\caption{Hash Join}
	\label{fig:join}
\end{figure}

Turning our attention to Figure~\ref{fig:join}, we find that the hash join provides
$17\%$ improvement in Writes and $50\%$ in Cycles for Q16. Similarly, for Q19, it provides
savings of about $42\%$ in Writes and $50\%$ in Cycles.

\begin{figure}[htbp]
\centering
\subfloat[Q13]{
  \includegraphics[height=29mm]{gb_q13.png}
}
%\hspace{0mm}
\subfloat[Q16]{
  \includegraphics[height=29mm]{gb_q16.png}
}
\caption{Group By}
\label{fig:gby}
\end{figure}

Finally, moving on to Figure~\ref{fig:gby}, the group-by operator in
Q16 provides Writes savings of $78\%$ and Cycles savings of $26\%$.
The hash table for group-by being much smaller than all DRAM variants for Q13, the
change in the DRAM size did not materially impact the Writes and
Cycles, consistently yielding the same improvement of 4\% and 1\% for the two
metrics, respectively.

Thus, we can conclude that the savings obtained by our algorithms are
not restricted to the environment where the DRAM is exclusively available
to one process. Instead, our algorithms can handle the unpredictability
in the DRAM availability due to multiple running processes equally well,
thereby being sufficiently robust to volatile system behaviour.

\subsection{Validating Write Estimators}
\label{validation}

We now move on to validating the estimators
(presented in Section~\ref{sort} through \ref{gby})  for the number of
writes incurred by the various database operators.

\subsubsection{Sort}
The size of the $orders$ table is approximately $214 \times 10^6$ bytes. The
conventional quicksort algorithm for this table in Q13 incurred writes of $233.6
\times 10^6$ words $= 934.4 \times 10^6 $bytes. Using Equation
\ref{eq:sort_conv}, the expected number of writes were: 

\begin{dmath}
\begin{small}
W_{sort\_conv}(bytes) = N_RL_R (0.5 \lceil log_2(\frac{N_R L_R}{D}) \rceil + 1) \\
\hspace*{100}= 214 \times 10^6 (0.5 \lceil log_2(\frac{214 \times 10^6}{4 * 10^6}) \rceil + 1) = 856 \times 10^6 
\end{small}
\end{dmath}

The PCM-conscious version used multi-pivot sort algorithm and
incurred writes of $110.6 \times 10^6$ words $= 441.8 \times 10^6 $
bytes. Replacing the values for $N_R L_R$ in Equation \ref{eq:mpivot},
we get the writes as 
\begin{small}
$$W_{mpivot}(bytes) \approx 2 \times 214 \times
10^6  = 428 \times 10^6 $$. 
\end{small}
Thus both the estimates are close to the
observed writes.



\subsubsection{Hash Join}
For the hash join in Q19, the values of $N_R$, $size_{entry}$, $J$,
$size_{j}$ are $2 \times 10^5$, $8$, $120$ and $8$, respectively. Here,
the observed writes for PCM-oblivious hash join were $0.90 \times 10^6$
words $= 3.6 \times 10^6$ bytes. Substituting the parameter values in Equation
\ref{eq:ht_conv}, we get the estimated writes as: 

\begin{dmath}
\begin{small}
W_{ht\_conv}(bytes) = N_R \times (size_{entry} + P) + J \times size_{j} \\
\hspace*{80}= 2 \times 10^5 (8+4) + 120 \times 8 \approx 2.4 \times 10^6 
\end{small}
\end{dmath}

Now, for the PCM-conscious version, the estimated writes are given by Equation \ref{eq:ht_pcm},
and substituting  the  parameter values, the writes are given by:  
\begin{small}
$$W_{ht\_pcm}(bytes) = 2 \times 10^5 (8-3) + 120 \times 8 \approx 10^6$$
\end{small}
which is close to the actual writes of $0.32 \times 10^6$ words $=
1.28 \times 10^6$ bytes.

\begin{table}[!h]                                                                                       
\centering                                                                                              
\caption{Validation of Write Estimators}
  \label{tab:estimator_validation}                                                                                
  %\centering                                                                                             
  \begin{small}                                                                                           
  \begin{tabular}{p{5cm}p{2cm}p{2cm}p{1cm}}
  \toprule                                                                                                
  
  \textbf{Operator} & \textbf{Esimated Writes} (e) & \textbf{Observed Writes} (o) & \textbf{Error Factor} $\frac{e-o}{o}$\\
  \midrule                                                                                                
  
  \textbf{Sort PCM-Oblivious} &  $856 \times 10^6$ & $934.4 \times 10^6$ & -0.08\\ 
    \textbf{Sort PCM-Conscious} &  $428 \times 10^6$ & $441.8 \times 10^6$ & -0.03\\ 
  \textbf{Hash Join PCM-Oblivious} &  $ 2.4 \times 10^6$ &$3.6 \times 10^6$ & -0.33\\ 
  \textbf{Hash Join PCM-Conscious} &  $1 \times 10^6$ & $1.28 \times 10^6$ & -0.22\\ 
  \textbf{Group-By PCM-Oblivious} &  $9.4 \times 10^6$ & $5.44 \times 10^6$ & 0.73\\ 
  \textbf{Group-By PCM-Conscious} &  $1.83 \times 10^6$ &$1.44 \times 10^6$ & 0.27\\ 
  
  \bottomrule                                                                                             
  \end{tabular}                                                                                           
  \end{small}                                                                                             
  \end{table} 
  
\subsubsection{Group-By}
The values of the parameters $N_R$, $L_R$, $P$, $G$ and $size_g$ for Q16
are $119056$, $48$, $4$, $18341$ and $48$, respectively.  The observed
writes for group-by for the PCM-oblivious algorithm was $1.36 \times 10^6$
words $= 5.44 \times 10^6$ bytes. Using Equation~\ref{eq:gby_conv_sort}
and replacing the parameter values, we get: 

\begin{dmath}
\begin{small}
W_{gb\_conv\_sort}(bytes) = N_RL_R (0.5 \lceil log_2(\frac{N_R L_R}{D}) \rceil + 1) + G \times size_g \\
\hspace*{90}= 119056 \times 48  (0.5 \lceil log_2(\frac{119056 \times 48}{4 \times 10^6}) \rceil + 1) + 18341 \times 48  
\approx 9.4 \times 10^6
\end{small}
\end{dmath}

Now, for the PCM-conscious version, which uses pointer-based hash grouping,
using Equation \ref{eq:gby_ptr_hash} results in:

\begin{dmath}
\begin{small}
W_{gb\_ptr\_hash}(bytes) = 2N_R \times P + G \times size_g
= 2 \times 119056\times 4 + 18341 \times 48
= 1.83 \times 10^6 
\end{small}
\end{dmath}

This closely corresponds to the observed writes of $0.36 \times 10^6$ words $= 1.44 \times 10^6$ bytes.




A summary of the above results is provided in
Table~\ref{tab:estimator_validation}. It is clear that our estimators
predict the write cardinality with a reasonable degree of accuracy for both the
PCM-oblivious and PCM-conscious algorithm, making them suitable for
incorporation in the query optimizer.



\end{appendix}