\section{Conclusion and Future Work}
\label{conclusion}
Designing database query execution algorithms for PCM platforms
requires a change in perspective from the traditional assumptions of
symmetric read and write overheads.  We presented here a variety of
modified algorithms for the workhorse database operators: \emph{sort},
\emph{join} and \emph{group-by}, which were constructed with a view
towards simultaneously reducing both the number of writes and the
overall query response time. Through detailed experimentation on benchmark
environments, we showed that substantial improvements on these metrics can
be obtained as compared to their contemporary PCM-oblivious counterparts.
Collaterally, the PCM cell lifetimes are also greatly extended by our
approaches.

Note that while our experiments were conducted on a PCM simulator, the
cycle-accurate nature of the simulator makes it likely that similar
performance will be exhibited in the real world as well. Moreover,
our write estimators facilitate the incorporation of these algorithms
in the query optimizer, as we have shown in our implementation.

Energy consumption is a major concern today in data centres since it directly relates to cost. We would therefore like to have a fresh look at optimizer design with respect to a revised metric \emph{performance/watt}. Given the major energy savings obtained from a PCM based hardware, the user might be willing to trade-off latency not only for the sake of writes but for the additional benefit of saving power cost. We see this as an interesting line of future work.

\begin{comment}
Even within the class of PCM conscious algorithms, there exist myriad
algorithm design choices offering varying degrees of writes and running
time. Nested loops join, for example, would incur the least amount of
writes for join when both relations fit in PCM, but might prove to be
extremely slow. We need to come up with metrics that can quantify this
trade-off based upon some measure of the lifetime that the PCM memory
module is expected to serve and the maximum delay the user is willing to
bear. These metrics too need to be integrated with the query optimizer
for it to make suitable plan choices for PCM based hardware. We see this
as an interesting line of future work.
\end{comment}
