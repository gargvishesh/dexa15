\section{Conclusion}
\label{conclusion}
Designing database query execution algorithms for PCM platforms requires a
change in perspective from the traditional assumptions of symmetric read
and write overheads.  We presented here a variety of minimally modified algorithms
for the workhorse database operators: \emph{sort}, \emph{join} and
\emph{group-by}, which were constructed with a view towards simultaneously
reducing both the number of writes and the response time. Through detailed
experimentation on complete TPC-H benchmark queries, we showed that
substantial improvements on these metrics can be obtained as compared
to their contemporary PCM-oblivious counterparts.  Collaterally, the
PCM cell lifetimes are also greatly extended by our approaches.

Using our write estimators for uniformly distributed data, we presented a
redesigned database optimizer, thereby incorporating PCM-consciousness in
all the layers of the database engine. We also presented initial results
showing how this can influence plan choices, and improve the write
performance by a very large extent.  Note that while our experiments
were conducted on a PCM simulator, the cycle-accurate nature of the
simulator makes it likely that similar performance will be exhibited in
the real world as well. In our future work, we would like to design
write estimators that leverage the metadata statistics to accurately
predict writes for skewed data. Additionally, we would like to come up with better multi-objective optimization algorithms for query plan selection.

Overall, the results of this paper augur well for an easy migration of
current database engines to leverage the benefits of tomorrow's PCM-based
computing platforms.
