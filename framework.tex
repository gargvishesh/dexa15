\section{Problem Framework}
\label{assumptions}
In this section, we overview the problem framework, the assumptions made
in our analysis, and the notations used in the sequel.

As mentioned in the Introduction, we model the \model{} memory
organization (Figure~\ref{fig:pcm_models} (c)). The DRAM buffer is of
size $D$ and is organized in a \emph{K-way set-associative} manner, like the higher level cache memories. Moreover, its operation is also identical to that of an \emph{inclusive} cache in the memory hierarchy; fetching a new DRAM line (256B in our experimental setup) from PCM each time there is a DRAM miss. The last level cache in turn fetches its data from the DRAM buffer.

A \textit{data-comparison write (DCW)} scheme \cite{write} is used for
the writing of PCM memory blocks during eviction from DRAM -- in this
scheme, the memory controller compares the existing PCM block to the newly
evicted DRAM block, and selectively writes back only the modified words.
Further, \textit{N-Chance}~\cite{nchance} is used as the DRAM eviction
policy -- here, the first clean way in the $N$ least recently used ways
(assuming $N$ is less than $K$, the DRAM cache associativity) in a set
is chosen as the eviction victim. Only when all the $N$ ways are dirty,
the LRU entry is evicted. We choose this scheme due to its preference
for evicting non-dirty entries over dirty candidates, thereby saving
on writes.

As described above, the simulator implements a realistic DRAM
buffer. However, in our write analyses and estimators, we assume for
tractability that there are no conflict misses in the DRAM. Thus, for
any operations dealing with data whose size is within the DRAM capacity,
our analysis assumes no evictions and consequently no writes.

With regard to the operators, we use $R$ to denote the input relation
for the \textit{sort} and \textit{group-by} unary operators.  Whereas,
for the binary \textit{hash join} operator, $R$ is used to denote the
smaller relation, on which the hash table is constructed, while $S$
denotes the probing relation. 

We assume that all the input relations are completely PCM-resident.
Further, for simplicity of presentation, we assume that the sort,
hash-join and group-by expressions are on single relational attributes --
the extension to multiple attributes is straightforward.

A summary of the main notation used in the analysis of the following
sections is provided in Table~\ref{tab:notations}.

\begin{table}[!h]
\centering
\caption{Notations Used in Operator Analysis}
\label{tab:notations}
\begin{small}
\begin{tabular}{p{2cm}p{9cm}}
\toprule  
\textbf{Term} & \textbf{Description}\\ 
\midrule
\textbf{$D$} & DRAM size\\
\textbf{$K$} & DRAM Associativity\\
\textbf{$P$} & Pointer size\\
\textbf{$N_R, N_S$} & Row cardinality of relations R and S, respectively\\
\textbf{$L_R, L_S$} & Tuple size of relations R and S, respectively\\
\textbf{$size_{entry}$} & Size of each hash table entry\\
\textbf{$J,G$} & Output tuple cardinalities of join and group-by operators, respectively\\
\textbf{$size_{j},size_{g}$} & Output tuple sizes of join and group-by operators, respectively\\
\bottomrule
\end{tabular}
\end{small}
\end{table}
