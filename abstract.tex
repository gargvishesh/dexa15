\begin{abstract} 

Phase Change Memory (PCM) is a new \emph{non-volatile} memory technology
that is comparable to traditional DRAM with regard to read latency,
and markedly superior with regard to storage density and idle power
consumption. Due to these desirable characteristics, PCM is expected
to play a significant role in the next generation of computing
systems. However, it also has limitations in the form of expensive
writes and limited write endurance. Accordingly, recent research 
has investigated how database engines may be redesigned to
suit DBMS deployments on the new technology.

In this paper, we address the pragmatic goal of designing PCM-conscious database operators
using ``off-the-shelf'' techniques for minimal integration effort with existing systems. Specifically, we target implementations of ``workhorse'' database operators: \emph{sort},
\emph{hash join} and \emph{group-by} that substantively improve write performance without compromising on execution times. We provide simple but effective \emph{estimators} of the writes incurred by these techniques,
and use them to redesign the database query optimizer cost model.
We test these techniques on \emph{end-to-end} TPC-H benchmark queries on writes, latency and wear distributions metrics. Our experimental results
suggest that the PCM-conscious operators collectively reduce the number
of writes by a factor of 2 to 3, while concurrently improving the query
response times by about 20 to 30\%. In essence, our algorithms provide both short-term and
long-term benefits. These outcomes augur well for database engines that
wish to leverage the impending transition to PCM-based computing.

\end{abstract}
