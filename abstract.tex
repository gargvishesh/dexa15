\begin{abstract} 

Phase Change Memory (PCM) is a new \emph{non-volatile} memory technology
that is comparable to traditional DRAM with regard to read latency,
and markedly superior with regard to storage density and idle power
consumption. Due to these desirable characteristics, PCM is expected
to play a significant role in the next generation of computing
systems. However, it also has limitations in the form of expensive
writes and limited write endurance. Accordingly, recent research 
has investigated how database engines may be redesigned to
suit DBMS deployments on the new technology.

\begin{comment}
In this paper, we explore the design of PCM-conscious database operators
for main memory organizations comprised of PCM augmented with a small
hardware-controlled DRAM buffer.  Specifically, we present modified
constructions of the ``workhorse'' database operators: \emph{sort},
\emph{hash join} and \emph{group-by}, that substantively improve the
write performance without compromising on execution times. We also
provide \emph{estimators} of the writes incurred by these techniques,
thereby facilitating integration with the database query optimizer.
\end{comment}

In this paper, we explore the design of PCM-conscious database operators
for main memory organizations comprised of PCM augmented with a small
hardware-controlled DRAM buffer. Specifically, we present modified
constructions of the ``workhorse'' database operators: \emph{sort},
\emph{hash join} and \emph{group-by}, which use existing but overlooked ``off-the-shelf'' algorithms for substantively improving the
write performance without compromising on execution times. We also
provide \emph{estimators} of the writes incurred by these techniques,
thereby facilitating integration with the database query optimizer.

The proposed techniques are implemented on a state-of-the-art
architectural simulator, which we extend to model PCM, and their
performance is assessed on a workload of TPC-H benchmark queries. Apart from
the number of writes and query response times, the algorithms are
also evaluated on their wear distributions. Our experimental results
suggest that the PCM-conscious operators collectively reduce the number
of writes by a factor of 2 to 3, while concurrently improving the query
response times by about 20 to 30\%.  Further, the wear distribution
is also appreciably smoother. In essence, our algorithms provide both short-term and
long-term benefits. These outcomes augur well for database engines that
wish to leverage the impending transition to PCM-based computing.

\end{abstract}
